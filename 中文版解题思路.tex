% !TEX program = xelatex
\documentclass[12pt, a4paper]{article}
\usepackage{ctex} % 中文支持
\usepackage{amsmath, amssymb, amsthm} % 数学公式支持
\usepackage{geometry} % 页面布局
\usepackage{graphicx} % 图片支持
\usepackage{booktabs} % 三线表
\usepackage{enumitem} % 列表控制
\usepackage{hyperref} % 超链接
\usepackage{titlesec} % 标题格式
\usepackage{fancyhdr} % 页眉页脚

% 页面设置
\geometry{left=2.5cm, right=2.5cm, top=2.5cm, bottom=2.5cm}

% 章节标题格式
\titleformat{\section}{\Large\bfseries}{\thesection}{1em}{}
\titleformat{\subsection}{\large\bfseries}{\thesubsection}{1em}{}

% 文档信息
\title{塔式太阳能光热电站年发电量最大化数学模型}
\author{您的姓名}
\date{\today}

\begin{document}

\maketitle

\begin{abstract}
本文建立了塔式太阳能光热电站年发电量最大化的数学模型。模型涵盖了坐标系定义、反射几何、镜面定向、太阳位置计算、能量通量分析以及效率因子等多个方面,为电站的优化设计和性能评估提供了理论基础。
\end{abstract}

\tableofcontents
\newpage

% ==================== 您提供的内容开始 ====================

\section{坐标系统与变量定义}

我们定义三维笛卡尔坐标系:
- 原点 \( O \):中央接收塔基座(定日镜场中心)
- \( z \)-轴:垂直向上;接收器孔径中心位于 \( T = (0, 0, H_t) \)
- 定日镜场:以原点为中心的圆形区域
- 定日镜 \( i \):中心坐标 \( M_i = (x_i, y_i, 6) \),镜面尺寸 \( 6 \times 6 \, \text{m} \)

太阳位置角:
- 太阳高度角 \( \alpha_s(t) \)(地平线以上的仰角)
- 太阳方位角 \( \gamma_s(t) \)(从北顺时针测量,0° 到 360°)

定日镜定向角:
- 仰角/倾斜角 \( \theta_i(t) \)(镜面法线与水平面夹角)
- 方位角 \( \phi_i(t) \)(镜面法线在水平面投影,从北顺时针测量)

\section{反射几何与镜面法向量}

从地球指向太阳的单位向量(太阳方向向量):
\[
\mathbf{\hat{s}}_{\text{sun}} = 
\begin{bmatrix}
\cos\alpha_s \sin\gamma_s \\
\cos\alpha_s \cos\gamma_s \\
\sin\alpha_s
\end{bmatrix}
\]
(约定:\( x \)-轴向东,\( y \)-轴向北,\( z \)-轴向上)

入射光线方向(从太阳到镜面)为 \( \mathbf{i} = -\mathbf{\hat{s}}_{\text{sun}} \)。

反射光线方向(从镜面到目标):
\[
\mathbf{\hat{r}} = \frac{T - M_i}{\| T - M_i \|}
\]

镜面单位法向量(使用反射定律,入射向量指向镜面):
\[
\mathbf{\hat{n}} = \frac{\mathbf{i} + \mathbf{\hat{r}}}{\| \mathbf{i} + \mathbf{\hat{r}} \|}
\]
这确保了 \( \mathbf{\hat{r}} = \mathbf{i} - 2(\mathbf{i} \cdot \mathbf{\hat{n}})\mathbf{\hat{n}} \) 成立。

从 \( \mathbf{\hat{n}} = (n_x, n_y, n_z) \),镜面定向角为:
\[
\theta_i = \arcsin(n_z), \quad \phi_i = \operatorname{atan2}(n_x, n_y)
\]
(使用 atan2 获得正确象限,结果以弧度表示,从北顺时针)

\section{有效面积与余弦损失}

镜面面积:\( A_m = 36 \, \text{m}^2 \)。

余弦效率(朝向太阳的投影面积):
\[
\eta_{\cos} = \mathbf{\hat{n}} \cdot (-\mathbf{\hat{s}}_{\text{sun}}) = \mathbf{\hat{n}} \cdot \mathbf{i}
\]
这是镜面法线与入射阳光夹角的余弦值。

\section{太阳位置模型}

太阳赤纬(日数 \( n \),1 到 365):
\[
\delta \approx 23.45^\circ \times \sin\left( \frac{2\pi (n - 80)}{365} \right) \quad \text{(单位:度)}
\]

时角:
\[
h = 15^\circ \times (\text{太阳时} - 12)
\]
(太阳时 = 地方时 + 时差校正;为简化,时差可忽略)

太阳高度角:
\[
\sin \alpha_s = \sin \varphi \, \sin \delta + \cos \varphi \, \cos \delta \, \cos h
\]
太阳方位角(从北顺时针):
\[
\cos \gamma_s = \frac{\sin \delta \cos \varphi - \cos \delta \sin \varphi \cos h}{\cos \alpha_s}, 
\quad \sin \gamma_s = \frac{\cos \delta \sin h}{\cos \alpha_s}
\]
\[
\gamma_s = \operatorname{atan2}(\sin \gamma_s, \cos \gamma_s) \quad \text{(需要时调整象限)}
\]

\section{能量通量与效率因子}

\textbf{法向直接辐照度 (DNI)} \( I_{\text{dni}}(t) \):\\
简化的晴空模型(如 Hottel 模型)或使用典型气象年 (TMY) 数据。

\textbf{反射率}:\( \rho \approx 0.88 \)(典型镜面反射率)。

\textbf{大气衰减}(镜面与接收器之间):
\[
\eta_{\text{atm}} = a^{d}, \quad a \approx 0.99 \ \text{每公里}, \quad d = \| M_i - T \| / 1000 \ \text{(公里)}
\]

\textbf{截断效率} \( \eta_{\text{trunc}} \):反射锥进入接收器孔径的比例。取决于镜面形状、瞄准、距离和接收器尺寸。简化处理:如果接收器完全捕获反射光束,\( \eta_{\text{trunc}} = 1 \);否则通过锥截计算。

\textbf{遮挡/阴影损失} \( \eta_{\text{block}} \):此处忽略(=1)以简化,但在详细布局优化中必须建模。

定日镜 \( i \) 的瞬时反射功率:
\[
P_i(t) = I_{\text{dni}}(t) \cdot A_m \cdot \rho \cdot \eta_{\cos} \cdot \eta_{\text{atm}} \cdot \eta_{\text{trunc}}
\]

场总热功率:
\[
P_{\text{field}}(t) = \sum_{i=1}^{N} P_i(t)
\]

\section{年发电量计算}

热电转换效率:\( \eta_{\text{th}} \approx 0.35 \)。

年发电量:
\[
E_{\text{year}} = \eta_{\text{th}} \int_{\text{year}} P_{\text{field}}(t) \, dt
\]
在白天(\( \alpha_s > 0 \))且定日镜不在维护/夜间时段积分。

年平均电功率:
\[
\overline{P}_{\text{electric}} = \frac{E_{\text{year}}}{8760 \ \text{小时}}
\]

\section{最大化方法}

给定固定的定日镜位置,唯一控制是瞄准定律(已由反射几何设定)。\\
年输出最大化通过以下方式实现:
- 优化布局以最小化平均衰减和余弦损失
- 通过适当的接收器尺寸确保 \( \eta_{\text{trunc}} \approx 1 \)
- 在此模型中,我们计算所有 \( t \) 的 \( P_{\text{field}}(t) \),然后积分

一旦坐标和瞄准策略固定,就没有额外的自由度。

% ==================== 您提供的内容结束 ====================

\section*{模型验证与讨论}

该模型为塔式太阳能电站的优化设计提供了完整的数学框架。在实际应用中,还需要考虑:
\begin{itemize}
    \item 定日镜场的布局优化算法
    \item 接收器的热损失模型
    \item 实际运行中的维护周期和效率衰减
    \item 不同气候条件下的适应性分析
\end{itemize}

\end{document}